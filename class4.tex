\documentclass{report}
\usepackage[utf8]{inputenc}
\usepackage{graphicx}
\usepackage{devanagari}
\usepackage{subcaption}

\title{figures n beamer}
\author{bikash.chawal }
\date{June 2020}

\begin{document}
\listoffigures
\maketitle

\chapter{Introduction}
Bhaktapur Durbar Square, (Nepal Bhasa: Devanagari :),the royal palace of the old Bhaktapur Kingdom, 1,400 metres (4,600 ft) above sea level.[1] It is a UNESCO World Heritage Site.
\begin{figure}[ht]
    \centering
    \includegraphics[width=1\textwidth]{kalinchowk.jpg}
    \caption{Bhaktapur Durbar Square}
    \label{dsquare}
\end{figure}
The fig \ref{dsquare} shows the palace courtyard.The Bhaktapur Durbar Square is located in the current town of Bhaktapur, also known as Khwopa,[1] which lies 13 km east of Kathmandu. While the complex consists of at least four distinct squares (Durbar Square, Taumadhi Square, Dattatreya Square and Pottery Square),[2] the whole area is informally known as the Bhaktapur Durbar Square and is a highly visited site in the Kathmandu Valley.
\begin{figure}
  \centering
    \begin{subfigure}[b]{0.8\textwidth}
       \centering
    \includegraphics[width=0.5\textwidth]{kalinchowk.jpg}
    \caption{dsquare1}
    \label{d1}
    \end{subfigure}
    %\centering
    \begin{subfigure}[b]{0.8\textwidth}
     \centering
    \includegraphics[width=0.5\textwidth]{kalinchowk.jpg}
    \caption{dsquare2}
    \label{d2}
    \end{subfigure}
    \caption{sub figure}
\end{figure}



\end{document}
